\chapter{Introdução}

A era da informação transformou o mundo dos negócios. Hoje, as empresas estão cercadas por uma imensidão de dados, oriundos de diversas fontes e em formatos variados. A capacidade de coletar, analisar e usar esses dados tornou-se um fator decisivo na competitividade e sucesso de uma organização. Esta é uma realidade que se aplica a todos os setores da indústria, em que a tomada de decisões orientada por dados passou a ser um aspecto crucial para otimizar operações, identificar oportunidades de mercado, personalizar ofertas e melhorar o atendimento ao cliente.

Entretanto, dados brutos por si só têm pouco valor. Para que se tornem informações úteis e acionáveis, os dados precisam passar por um processo de extração, transformação e ingestão (ETL). A extração envolve a coleta de dados de diferentes fontes, que podem variar desde bancos de dados internos até mídias sociais e sites da web. A transformação envolve o processamento desses dados para torná-los adequados para análise, o que pode incluir a limpeza de dados, a integração de dados de várias fontes, a conversão de formatos de dados e a criação de agregações. Por fim, a ingestão envolve o carregamento dos dados transformados em um sistema de destino, geralmente um data warehouse ou data lake, onde eles podem ser acessados para análise e relatórios.

O processo de ETL é, portanto, fundamental para a eficácia da tomada de decisões orientada por dados. A arquitetura de solução em nuvem surge como uma solução para muitos desafios associados ao ETL, oferecendo flexibilidade, escalabilidade e eficiência na coleta, processamento e armazenamento de grandes volumes de dados. Além disso, a natureza on-demand da computação em nuvem ajuda as empresas a reduzirem os custos operacionais e de infraestrutura, ao mesmo tempo em que acelera o tempo de insight dos dados. Assim, uma arquitetura de solução em nuvem para processos de ETL pode trazer vantagens significativas para empresas que buscam uma abordagem mais orientada a dados em sua tomada de decisões.


Nesse contexto, este trabalho apresenta o desenvolvimento de uma arquitetura de solução em nuvem para a coleta e ingestão de dados cadastrais de CNPJs brasileiros. A escolha da arquitetura em nuvem para este projeto se baseia na necessidade de uma solução com baixo custo, alta eficiência e praticidade, capaz de lidar com grandes volumes de dados e proporcionar uma visão analítica para tomada de decisões estratégicas mais eficazes.
