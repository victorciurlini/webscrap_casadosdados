\section{Embasamento Teórico}

\subsection{Extração, Transformação e Carregamento (ETL)}

A Extração, Transformação e Carregamento (ETL) é uma parte fundamental de qualquer sistema de informação. Ela é responsável por coletar dados de várias fontes, transformá-los para se adequar ao modelo de dados do sistema de destino e, finalmente, realizar a ingestão dos dados extraídos e tratados no sistema de destino \cite{simitsis2005optimizing}. 

No contexto do projeto em questão, o ETL é usado para coletar dados de CNPJs abertos diariamente no Brasil, transformá-los e carrega-los em um banco de dados relacional. Este processo é crucial para garantir que os times de prospecção de clientes e de análise de dados tenham acesso a informações de forma organizada.

A otimização dos processos de ETL é uma área de pesquisa ativa. Simitsis et al. \cite{simitsis2005optimizing} discutem várias técnicas para otimizar processos de ETL em Data warehouses. Embora o projeto em questão não envolva um  Data warehouse, muitas das técnicas discutidas por Simitsis et al. podem ser aplicadas para otimizar o processo de ETL.

Um aspecto importante do ETL é a capacidade de lidar com grandes volumes de dados. Ong et al. \cite{ong2017dynamic} discutem um enfoque híbrido para processos de ETL que pode lidar com grandes volumes de dados de saúde. Embora os dados discutidos por Ong et al. sejam específicos para o domínio da saúde, as técnicas e abordagens que eles discutem podem ser aplicadas a outros domínios, incluindo o domínio do projeto em questão.

Em resumo, o ETL é uma parte crucial do projeto em questão. Ele permite a coleta, transformação e carregamento de dados de CNPJs abertos diariamente no Brasil, garantindo que os times de prospecção de clientes e de análise de dados tenham acesso a informações atualizadas e estruturadas. A pesquisa em otimização de processos de ETL e em ETL para grandes volumes de dados fornece técnicas e abordagens úteis que podem ser aplicadas para melhorar o processo de ETL no projeto.




\subsection{Solução em Nuvem}

A computação em nuvem é uma solução amplamente adotada para hospedar e executar aplicativos em um ambiente virtualizado, fornecendo recursos de armazenamento e processamento sob demanda. Uma das vantagens-chave da computação em nuvem é a escalabilidade, que permite dimensionar verticalmente ou horizontalmente os recursos de acordo com as necessidades da aplicação \cite{fernando2016mobile}. Isso proporciona um desempenho aprimorado, permitindo que as aplicações respondam rapidamente às demandas dos usuários e garantindo uma experiência satisfatória.

No contexto do projeto em questão, a adoção de uma solução em nuvem oferece benefícios significativos em termos de desempenho. A capacidade de dimensionamento automático dos recursos da nuvem permite que a aplicação se adapte dinamicamente ao aumento ou diminuição da carga de trabalho, garantindo um desempenho consistente mesmo durante os períodos de pico de demanda. Além disso, a computação em nuvem oferece acesso a uma infraestrutura global distribuída, permitindo que os dados sejam armazenados e processados em servidores localizados próximos aos usuários finais, reduzindo a latência e melhorando a velocidade de resposta \cite{fernando2016mobile}.

Outro fator importante a ser considerado é o custo das aplicações em nuvem. Ao adotar uma solução em nuvem, as empresas podem se beneficiar de modelos de preços flexíveis, pagando apenas pelos recursos utilizados. Isso elimina a necessidade de investimentos antecipados em infraestrutura física e permite que as empresas reduzam seus custos operacionais. Além disso, a natureza virtualizada da computação em nuvem permite que os recursos sejam compartilhados entre várias aplicações, otimizando a utilização e reduzindo ainda mais os custos \cite{fernando2016mobile}.

No entanto, é importante considerar alguns desafios relacionados ao desempenho e ao custo das aplicações em nuvem. O desempenho pode ser afetado por fatores como a latência da rede, a sobrecarga dos servidores e a alocação inadequada de recursos. Portanto, é essencial realizar uma análise cuidadosa dos requisitos da aplicação e configurar adequadamente os recursos da nuvem para garantir um desempenho ótimo \cite{fernando2016mobile}. Quanto ao custo, é necessário acompanhar de perto os gastos com os recursos da nuvem e otimizar continuamente sua utilização para evitar custos excessivos \cite{fernando2016mobile}.

No geral, a computação em nuvem oferece uma solução flexível e escalável para hospedar e executar aplicações, proporcionando um desempenho aprimorado e reduzindo os custos operacionais. No contexto do projeto em questão, a adoção de uma solução em nuvem permitirá que a aplicação seja dimensionada conforme necessário, garantindo um desempenho consistente e eficiente, ao mesmo tempo em que oferece uma estrutura de custos mais vantajosa para a empresa.
